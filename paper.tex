% Options for packages loaded elsewhere
% Options for packages loaded elsewhere
\PassOptionsToPackage{unicode}{hyperref}
\PassOptionsToPackage{hyphens}{url}
\PassOptionsToPackage{dvipsnames,svgnames,x11names}{xcolor}
%
\documentclass[
]{article}
\usepackage{xcolor}
\usepackage{amsmath,amssymb}
\setcounter{secnumdepth}{5}
\usepackage{iftex}
\ifPDFTeX
  \usepackage[T1]{fontenc}
  \usepackage[utf8]{inputenc}
  \usepackage{textcomp} % provide euro and other symbols
\else % if luatex or xetex
  \usepackage{unicode-math} % this also loads fontspec
  \defaultfontfeatures{Scale=MatchLowercase}
  \defaultfontfeatures[\rmfamily]{Ligatures=TeX,Scale=1}
\fi
\usepackage{lmodern}
\ifPDFTeX\else
  % xetex/luatex font selection
\fi
% Use upquote if available, for straight quotes in verbatim environments
\IfFileExists{upquote.sty}{\usepackage{upquote}}{}
\IfFileExists{microtype.sty}{% use microtype if available
  \usepackage[]{microtype}
  \UseMicrotypeSet[protrusion]{basicmath} % disable protrusion for tt fonts
}{}
\makeatletter
\@ifundefined{KOMAClassName}{% if non-KOMA class
  \IfFileExists{parskip.sty}{%
    \usepackage{parskip}
  }{% else
    \setlength{\parindent}{0pt}
    \setlength{\parskip}{6pt plus 2pt minus 1pt}}
}{% if KOMA class
  \KOMAoptions{parskip=half}}
\makeatother
% Make \paragraph and \subparagraph free-standing
\makeatletter
\ifx\paragraph\undefined\else
  \let\oldparagraph\paragraph
  \renewcommand{\paragraph}{
    \@ifstar
      \xxxParagraphStar
      \xxxParagraphNoStar
  }
  \newcommand{\xxxParagraphStar}[1]{\oldparagraph*{#1}\mbox{}}
  \newcommand{\xxxParagraphNoStar}[1]{\oldparagraph{#1}\mbox{}}
\fi
\ifx\subparagraph\undefined\else
  \let\oldsubparagraph\subparagraph
  \renewcommand{\subparagraph}{
    \@ifstar
      \xxxSubParagraphStar
      \xxxSubParagraphNoStar
  }
  \newcommand{\xxxSubParagraphStar}[1]{\oldsubparagraph*{#1}\mbox{}}
  \newcommand{\xxxSubParagraphNoStar}[1]{\oldsubparagraph{#1}\mbox{}}
\fi
\makeatother


\usepackage{longtable,booktabs,array}
\usepackage{calc} % for calculating minipage widths
% Correct order of tables after \paragraph or \subparagraph
\usepackage{etoolbox}
\makeatletter
\patchcmd\longtable{\par}{\if@noskipsec\mbox{}\fi\par}{}{}
\makeatother
% Allow footnotes in longtable head/foot
\IfFileExists{footnotehyper.sty}{\usepackage{footnotehyper}}{\usepackage{footnote}}
\makesavenoteenv{longtable}
\usepackage{graphicx}
\makeatletter
\newsavebox\pandoc@box
\newcommand*\pandocbounded[1]{% scales image to fit in text height/width
  \sbox\pandoc@box{#1}%
  \Gscale@div\@tempa{\textheight}{\dimexpr\ht\pandoc@box+\dp\pandoc@box\relax}%
  \Gscale@div\@tempb{\linewidth}{\wd\pandoc@box}%
  \ifdim\@tempb\p@<\@tempa\p@\let\@tempa\@tempb\fi% select the smaller of both
  \ifdim\@tempa\p@<\p@\scalebox{\@tempa}{\usebox\pandoc@box}%
  \else\usebox{\pandoc@box}%
  \fi%
}
% Set default figure placement to htbp
\def\fps@figure{htbp}
\makeatother





\setlength{\emergencystretch}{3em} % prevent overfull lines

\providecommand{\tightlist}{%
  \setlength{\itemsep}{0pt}\setlength{\parskip}{0pt}}



 


\makeatletter
\@ifpackageloaded{caption}{}{\usepackage{caption}}
\AtBeginDocument{%
\ifdefined\contentsname
  \renewcommand*\contentsname{Table of contents}
\else
  \newcommand\contentsname{Table of contents}
\fi
\ifdefined\listfigurename
  \renewcommand*\listfigurename{List of Figures}
\else
  \newcommand\listfigurename{List of Figures}
\fi
\ifdefined\listtablename
  \renewcommand*\listtablename{List of Tables}
\else
  \newcommand\listtablename{List of Tables}
\fi
\ifdefined\figurename
  \renewcommand*\figurename{Figure}
\else
  \newcommand\figurename{Figure}
\fi
\ifdefined\tablename
  \renewcommand*\tablename{Table}
\else
  \newcommand\tablename{Table}
\fi
}
\@ifpackageloaded{float}{}{\usepackage{float}}
\floatstyle{ruled}
\@ifundefined{c@chapter}{\newfloat{codelisting}{h}{lop}}{\newfloat{codelisting}{h}{lop}[chapter]}
\floatname{codelisting}{Listing}
\newcommand*\listoflistings{\listof{codelisting}{List of Listings}}
\makeatother
\makeatletter
\makeatother
\makeatletter
\@ifpackageloaded{caption}{}{\usepackage{caption}}
\@ifpackageloaded{subcaption}{}{\usepackage{subcaption}}
\makeatother
\usepackage{bookmark}
\IfFileExists{xurl.sty}{\usepackage{xurl}}{} % add URL line breaks if available
\urlstyle{same}
\hypersetup{
  pdftitle={Comparison of Control and Treatment Groups},
  pdfauthor={Your Name},
  pdfkeywords={insect physiology, treatment effects, statistical
analysis},
  colorlinks=true,
  linkcolor={blue},
  filecolor={Maroon},
  citecolor={Blue},
  urlcolor={Blue},
  pdfcreator={LaTeX via pandoc}}


\title{Comparison of Control and Treatment Groups}
\author{Your Name}
\date{2026-02-16}
\begin{document}
\maketitle
\begin{abstract}
This study compares measurements between control and treatment groups
across two measurement dates. We performed t-tests and linear regression
analysis to evaluate differences. Treatment group exhibited
significantly elevated values compared to control group.
\end{abstract}


\section{Introduction}\label{introduction}

Understanding differences between experimental conditions is fundamental
to scientific research. This analysis compares measurements collected
from control and treatment groups to evaluate whether the experimental
intervention produces measurable changes. We hypothesized that the
treatment would result in elevated measurements compared to controls.

\section{Methods}\label{methods}

\subsection{Study Design and Data}\label{study-design-and-data}

We collected measurements from both control and treatment groups on two
separate dates (January 15 and January 20, 2024). A total of 20
observations were recorded (10 per group, with measurements repeated
across two dates). One measurement in the control group was excluded due
to missing data.

\subsection{Statistical Analysis}\label{statistical-analysis}

We employed two complementary approaches to evaluate group differences:

\begin{enumerate}
\def\labelenumi{\arabic{enumi}.}
\item
  \textbf{T-test}: A two-sample t-test compared mean values between
  groups, with statistical significance assessed at α = 0.05.
\item
  \textbf{Linear regression}: A simple linear model with group as the
  predictor was fit to the data to quantify the magnitude of the
  treatment effect.
\end{enumerate}

\section{Results}\label{results}

\subsection{Descriptive Statistics}\label{descriptive-statistics}

The treatment group showed consistently higher measurements compared to
controls:

\begin{longtable}[]{@{}lrrrrr@{}}
\caption{Summary statistics by group}\tabularnewline
\toprule\noalign{}
group & n & mean & sd & min & max \\
\midrule\noalign{}
\endfirsthead
\toprule\noalign{}
group & n & mean & sd & min & max \\
\midrule\noalign{}
\endhead
\bottomrule\noalign{}
\endlastfoot
Control & 9 & 24.22 & 0.78 & 23.1 & 25.3 \\
Treatment & 10 & 29.03 & 0.77 & 27.9 & 30.2 \\
\end{longtable}

Mean values were 24.22 for the control group and 29.03 for the treatment
group, representing a difference of 4.81 units.

\subsection{Statistical Tests}\label{statistical-tests}

The t-test comparing groups revealed a significant difference (\emph{t}
= -13.556, \emph{p} = 0). The linear regression model confirmed this
finding, with the treatment effect coefficient of 4.808 units.

\subsection{Visualizations}\label{visualizations}

\pandocbounded{\includegraphics[keepaspectratio]{figures/boxplot.pdf}}

\pandocbounded{\includegraphics[keepaspectratio]{figures/timeseries.pdf}}

\textbf{Figure 1} displays the distribution of measurements by group,
illustrating the separation between control and treatment responses.
Individual points are overlaid on the boxplots to show data density.
\textbf{Figure 2} shows how mean values changed over the measurement
dates, with both groups maintaining consistent differences across time
points.

\section{Discussion}\label{discussion}

Our analysis provides strong evidence that the treatment significantly
increases measured values compared to controls. The effect is both
statistically significant and practically meaningful, with treatment
values approximately 5 units higher than controls on average.

The consistency of this effect across two measurement dates suggests a
robust treatment response rather than a transient effect. These findings
align with the theoretical expectation that the treatment would produce
elevated measurements and support the efficacy of the intervention.

Future work should investigate the mechanism underlying this treatment
effect and explore dose-response relationships to optimize the
intervention.

\section{Conclusion}\label{conclusion}

This study demonstrates a clear and statistically significant difference
between control and treatment groups, with the treatment producing
elevated measurements. The results provide evidence supporting the
effectiveness of the experimental intervention.

\section{References}\label{references}




\end{document}
